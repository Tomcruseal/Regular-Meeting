\documentclass{beamer}
\usepackage[UTF8,space,hyperref]{ctex}
\usepackage{graphicxsp}
\usetheme{}
\author{Tomcruseal}
\title{Regular Meeting Week 4}
\institute{School of Computer Science and Technology SCUT }


\begin{document}
	
	\frame{\titlepage}
	



    
    \begin{frame}论文分享
    	\begin{itemize}
    		\item Report from workshop and panel on the Status of
    		Serverless Computing and Function-as-a-Service
    		(FaaS) in Industry and Research
    	\end{itemize}
        
    \end{frame}
    
    \begin{frame}
    	\begin{itemize}
    		\item The authors believe that serverless computing is not only an exciting platform for researchers
    		to explore but also for academia to use.
    		%\item 
    	
        \end{itemize}
    \end{frame}

    \begin{frame}Serverless 和FaaS的基本概念
        \begin{itemize}
        	\item 起源,P2P
        	\item 在云环境中的定义,SaaS、Google App Engine
        	\item 最新的Serverless解决方案:really server-hidden,built to host functions,hide run and scale
        	\item run on-demand scale on-demand
        	
        \end{itemize}
    \end{frame}

    \begin{frame}IBM的定义
    	\begin{itemize}
    	    \item A cloud-native platform
    	    \item For short-running, stateless computation
    	    \item And event-driven applications
    	    \item which scales up and down instantly and
    	    automatically
    	    \item And charges for actual usage at a
    	    millisecond granularity
    	\end{itemize}
        
    \end{frame}

    \begin{frame}serverless的优点\\
        serverless擅长:短时运行、无状态、事件驱动
        \begin{itemize}
        	\item 微服务
        	\item 移动后端
        	\item 机器人、机器学习推断
        	\item 物联网
        	\item 适度的流处理
        	\item 服务集成
        \end{itemize}        
    \end{frame}

    \begin{frame}serverless的缺点\\
        serverless不擅长:长时运行、有状态、大量的数字计算
       \begin{itemize}
       	\item 数据库
       	\item 深度学习训练
       	\item 高负载的流分析
       	\item Spark/Hadoop分析
       	\item 数值模拟
       	\item 视频流
       \end{itemize}
        
    \end{frame}

    \begin{frame}
    	\begin{itemize}
    		\item 突发性的、计算密集的负载(如社交软件中的视频、图片的处理)
    		\item 后端:可实现无服务器后端,无需考虑弹性扩展、备份冗余,降低运维成本
    		\item 周期性/计划性的庞大数据量的处理
    		
    	\end{itemize}
        No server is better to manage than no server
    \end{frame}

    \begin{frame}基于FaaS的MapReduce--Pywren
        Pywren lets you run your existing python code at massive scale via AWS Lambda
        \begin{figure}
        	\centering
        	\includegraphics[height=2.7in]{3.png}
        \end{figure}
    \end{frame}

    \begin{frame}Serverless和FaaS的关系
    	\begin{figure}
    		\centering
    		\includegraphics[height=2.5in]{4.png}
    	\end{figure}
    \end{frame}

    \begin{frame}What is new about Serverless?
    	
        云生态环境提供了一系列中间件和人工智能服务,使得FaaS可以应用到
        \begin{itemize}
        	\item NLP
        	\item image recognition
        	\item manage state
        	\item record and monitor logs
        	\item send alerts
        	\item trigger events
        	\item perform authentication and authorization
        \end{itemize}
    \end{frame}

    \begin{frame}Stateless是否是不可或缺的?
    	
        将状态存储在FaaS之外使大数据触发多种微服务调用成为可能(Spark)
    \end{frame}
    
    \begin{frame}Can serverless work for longer running tasks?
    	
        目前,阿里云函数计算、AWS Lambda的函数最大运行时间均被限制到300秒以内。
        Alternative:
        \begin{itemize}
        	\item manage long running flows by combining multiple (small) FaaS invocations
        	\item handoff long running jobs to a different container service.
        \end{itemize}
    \end{frame}

    \begin{frame}编程模型
    	
        FaaS和大数据编程环境的结合(Spark,Flink,Hadoop,Storm,Heron)
        event-based programming
    \end{frame}
    
    \begin{frame}吹了这么多,那么Serverless和FaaS有什么缺点呢?
        \begin{itemize}
        	\item 使用docker、OpenStack时的扩展性问题
        	\item Serverless的计价方式
        	\item SLA
        	\item QoS
        	\item ...
        \end{itemize}
    \end{frame}
    
    \begin{frame}Current Serverless Systems
        \begin{itemize}
        	\item OpenWhisk
        	\item Google Cloud Functions,AWS Lambda,Azure Functions
        	\item 腾讯云无服务器云函数,阿里云函数计算
        	\item Pipsqueak,OpenLambda
        \end{itemize}
    \end{frame}

    \begin{frame}Can serverless help with scientific research?
    	
        Do have major importance for science and engineering research!
        \begin{itemize}
        	\item science data management
        %	\item 
        \end{itemize}
    \end{frame}
    
    \begin{frame}Future: What are low hanging fruits for serverless?
    	\begin{itemize}
    		\item applied to general purpose computing
    		\item 5 minute kill limit disappear
    		\item extend the MapReduce use of FaaS
    	\end{itemize}
        
    \end{frame}

    \begin{frame}
        \centering Thank you!
    \end{frame}

\end{document}