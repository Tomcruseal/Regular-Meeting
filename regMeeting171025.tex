\documentclass{beamer}
\usepackage[UTF8,space,hyperref]{ctex}
\usepackage{graphicxsp}
\usetheme{}
%\usepackage{listings}
\author{李智博}
\title{Regular Meeting Week 8}
\institute{School of Computer Science and Technology SCUT }


\begin{document}

	\frame{\titlepage}
	



   
    \begin{frame}基于腾讯无服务器云函数SCF的demo
    	\begin{itemize}
    		\item 无服务器云函数(Serverless Cloud Function)是腾讯云提供的无服务器(serverless)执行环境,用户无需购买和管理服务器,而只需使用平台支持的语言编写核心代码并设置代码运行的条件,代码即可在腾讯云基础设施上弹性、安全地运行。腾讯云完全管理底层计算资源,包括服务器 CPU、内存、网络和其他配置资源维护、代码部署、弹性伸缩、负载均衡等
    	\end{itemize}
        
    \end{frame}
    
    \begin{frame}
    	\begin{itemize}
    		\item 应用场景:视频应用、社交应用等场景下,用户上传的图片、音视频的总量大频率高,对移动应用的实时性和并发能力都有较高的要求
        \end{itemize}
    \end{frame}

    \begin{frame}构思:基于示例代码,利用无服务器函数处理图片
        \begin{itemize}
        	\item 用户上传图片至对象存储(COS)
        	\item 云函数从对象存储处下载图片并处理
        	\item 返回处理后的图片至对象存储
        \end{itemize}
    \end{frame}

    \begin{frame}实现流程
    	\begin{itemize}
    	    \item 创建函数与COS bucket的事件源映射
    	    \item 将对象上传到 COS 中的bucket(对象创建事件,可作为触发器trigger)
    	    \item COS Bucket检测到对象创建事件
    	    \item COS 调用函数并将事件数据(event)作为参数传递给函数
    	    \item SCF 平台接收到调用请求,执行函数
    	    \item 函数通过收到的事件数据获得了 Bucket 名称和文件名称,从该源 Bucket中获取该文件,使用图形库(PIL)处理图片,然后将其保存到目标Bucket上
    	\end{itemize}
    \end{frame}

    \begin{frame}COS bucket准备
        
        \begin{itemize}
        	\item 新建源 COS bucket
        	\item 新建目标 COS bucket
        	\item 上传源文件
        \end{itemize}
         \begin{figure}
         	\centering
         	\includegraphics[height=0.9in]{bucketList.png}
         \end{figure}
          \begin{figure}
         	\centering
         	\includegraphics[width=3.2in]{figureList.png}
         \end{figure}       
    \end{frame}
\begin{frame}[fragile]创建部署程序包
	\begin{itemize}
		\item main\_handler
		main\_handler用于处理事件
	\end{itemize}
	
	
	\begin{verbatim}
	def main_handler(event, context):
	\end{verbatim}
	
	
	\begin{verbatim}
	bucket = record['cos']['cosBucket']['name']
	cosobject = record['cos']['cosObject']['key']
	cosobject = cosobject.replace("/"+str(appid)+"/"+bucket,"")
	key = urllib.unquote_plus(cosobject.encode('utf8'))
	\end{verbatim}
\end{frame}

\begin{frame}[fragile]
\begin{itemize}
	\item 确定COS中的下载路径、之后的上传路径
	
\end{itemize}


\begin{verbatim}
download_path = '/tmp/{}{}'.format(uuid.uuid4(), key.split('/')[-1])
upload_path = '/tmp/resized-{}'.format(key.split('/')[-1])
\end{verbatim}
\end{frame}

\begin{frame}[fragile]

\begin{verbatim}
request = DownloadFileRequest(bucket, key, download_path)
downloadFileRet = cos_client.download_file(request)
\end{verbatim}
\end{frame}
    \begin{frame}[fragile]
      \begin{itemize}
	    \item 利用Python的Pillow库构建图片处理函数
      \end{itemize}


      \begin{verbatim}
        def processImage(origionalpath, resultpath):
          with Image.open(origional$backslash$path) as image:
          image.convert('L')
          image.save(resultpath)
      \end{verbatim}

\end{frame}

    \begin{frame}在控制台创建函数并测试
    	
        \url{https://console.cloud.tencent.com/scf/index/1/detail/imageDonkey}
        \begin{figure}
        	\centering
        	\includegraphics[height=3.2in]{FassConsole.png}
        \end{figure}
    \end{frame}

    \begin{frame}
      \begin{itemize}
      	\item 通过控制台的事件模板制定输入参数()这里是对象存储中的图片
      	\item 目前腾讯云支持四种事件模板
      	\item 在本例中,选择“COS上传/删除文件事件模板”
      \end{itemize}
    \end{frame}

    \begin{frame}
      \begin{itemize}
      	\item 测试程序:
      	\item 执行完毕后,在目标bucket能查看到相应的输出
      \end{itemize}
      \begin{figure}
      	\centering
      	\includegraphics[height=5.2in]{FassResult.png}
      \end{figure}
    
    \end{frame}    

    \begin{frame}
        \centering Thank you!
    \end{frame}

\end{document}