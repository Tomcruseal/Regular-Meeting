\documentclass{beamer}
\usepackage[UTF8,space,hyperref]{ctex}
\usepackage{graphicxsp}

\usetheme{}
%\usepackage{listings}
\author{Tomcruseal}
\title{Regular Meeting Week 19}
\institute{School of Computer Science and Technology SCUT }


\begin{document}

	\frame{\titlepage} \centering 学期总结
	



   
    \begin{frame}本学期工作简介
    	\begin{itemize}
    		\item 完成论文...
    		\item 学习函数式编程兼一些scala
    		\item FaaS and serverless架构
    		\item actor模型
    		\item 论文阅读
    	\end{itemize}
    \end{frame}
    
    \begin{frame}完成论文...
    	\begin{itemize}
    		\item Improved Gaussian Process Regression Based on Dual kd-Tree Traversal
        \end{itemize}
    \end{frame}

    \begin{frame}函数式编程(functional programming)
        \begin{itemize}
        	\item functional vs imperative
        	\item category theory(范畴论) 
        	\item immutable state
        	\item $\lambda$ calculus
        \end{itemize}
    \end{frame}

    \begin{frame}functional vs imperative
    	\begin{itemize}
    	    \item 最直接的例子是编程实践中经常遇到的递归和循环。
    	    \item 通常的循环是命令式编程的例子,对于一个for循环,需要维持一个内部状态(如
    	            循环控制变量),这个状态在每一次执行后将被更新
    	    \item 递归是函数式编程的一个特点,如在二叉树的深度优先搜索中,遍历某节点$N$等价于
    	            遍历其子节点$N_L$and $N_R$
    	    \item 然而在实际应用场景中,递归的开销相当大,容易造成stack overflow,这是可以
    	            进行尾递归优化(编译器 || 程序员)
    	\end{itemize}
    \end{frame}

    \begin{frame}
        \begin{itemize}
        	\item functional data structure in scala: immutable List, Stream
        	\item functional error handling:我们不throw错误,而是返回Option
        \end{itemize}    
    \end{frame}
    
    \begin{frame}[fragile]$\lambda$ calculus
	    \begin{itemize}
		    \item expression vs statement
	    \end{itemize}
        \begin{verbatim}
            val = (x: Int) => x+2
            (fn [x] x+2)
        \end{verbatim}
        \begin{itemize}
        	\item $\alpha$ equivalence and $\beta$ reduction
        \end{itemize}
            $$\lambda x.x \equiv \lambda y.y$$ 
            $\beta$ 规约
        \begin{itemize}
        	\item 0的表示、True/False的表示
        \end{itemize}
        $$\lambda y.\lambda x.x$$
        $$\lambda xy.x$$
        \begin{itemize}
        	\item $\lambda$ 演算与图灵机模型是等价的(Turing Complete),可以在图灵机上模拟。
        \end{itemize}
    \end{frame}

    \begin{frame}
        \begin{itemize}
        	\item FaaS,函数即服务
        	\item serverless架构,即无服务器架构
        \end{itemize}
        FaaS和severless架构具有以下特点
        \begin{itemize}
            \item 事件驱动
            \item 客户无需管理服务器等基础设施
            \item 弹性、高可靠
        \end{itemize}
    \end{frame}
    
    \begin{frame}FaaS和Serverless架构的前景与挑战
        \begin{itemize}
        	\item 面向云环境的编程模型
        	\item 业务逻辑与操作逻辑的分离
        	\item 混合云
        	\item cost/performance
        \end{itemize}
    \end{frame}
    
    \begin{frame}应用场景
        \begin{itemize}
        	\item 科学计算(pywren,Supercomputing as a Service...)
        	\item web 应用(失联儿童回家,与人工智能API结合,在线图片、视频解码...)
        	\item 实时流式数据处理
        	\item 移动后端
        \end{itemize}
    \end{frame}

    \begin{frame}actor模型
         \begin{itemize}
        	\item actor模型是一种基于消息传递(mailbox)的分布式计算模型
        	\item 数据不可变
        	\item 状态
         \end{itemize}
    \end{frame}
        
    \begin{frame}论文阅读
        \begin{itemize}
        	\item storm
        	\item ...
        \end{itemize}
    \end{frame}
    
    \begin{frame}
        \centering Thank you!
    \end{frame}

\end{document}