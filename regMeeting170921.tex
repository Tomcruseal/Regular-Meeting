\documentclass{beamer}
\usepackage[UTF8,space,hyperref]{ctex}
\usepackage{graphicxsp}
\usetheme{}
\author{Tomcruseal}
\title{Regular Meeting Week 3}
\institute{School of Computer Science and Technology SCUT }


\begin{document}
	
	\frame{\titlepage}
	
	

    
    \begin{frame}广义N体问题
    	\begin{itemize}
    		\item 广义N体问题与大数据的结合——运用广义N体问题的求解方法优化高斯过程回归的求解过程
    	\end{itemize}
        $$\overline{f}_{*}=\textbf{k}_{*}^{T}(K+\sigma^2 I)^{-1}\textbf{y}$$
        基本思路:运用空间划分数据结构(采用的是kd树)这一广义N体的方法优化矩阵$K$的计算过程从而加速$K$的运算,在训练集数据量大、维度高的情况下加快预测效率
    \end{frame}
    
    \begin{frame}当前面临的主要问题:
    	\begin{itemize}
    		\item 求得的近似矩阵$\tilde{K}$对预测结果的正确性的影响
    		\item 近似矩阵$\tilde{K}$不一定是半正定矩阵(不满足Mercer条件),因此运用Cholesky分解求$(K+\sigma^2 I)$的逆会出现问题(无解),然而此时若使用LU分解则会损失一倍的效率。
    	
        \end{itemize}
    \end{frame}

    \begin{frame}可能的应对策略
        \begin{itemize}
        	\item 对$\tilde{K}$作低秩近似
        	\item 通过kd树的剪枝规则控制$\tilde{K}$的结构,使其成为半正定矩阵
        	\item 通过$\sigma$超参数的设置使$(K+\sigma^2 I)$成为可逆矩阵(模型选择?经验选取?)
        	\item 对$\tilde{K}$施加约束,利用凸优化的思想找到最合适的半正定矩阵(优化计算时间?)
        	\item 不管,求解$(K+\sigma^2 I)$的广义逆矩阵(不确定有没有比Cholesky更快且稳定的求解方法)
        \end{itemize}
    \end{frame}

    \begin{frame}一些不着边际的想法
    	\begin{itemize}
    	    \item 由于scikit-learn是基于内存的计算,当矩阵规模超过一定数值是会产生“MemoryError”错误。原因:numpy数组中,一个float64占用8字节,因此$1000000\times100$的矩阵会占用约800MB的内存。考虑使用数据库进行优化(pytables)
    	    \item 核函数的选择,后期可以试用不同的核函数、选取不同的参数
    	    \item 对于适当的数据集会有很好的效果	
    	\end{itemize}
        
    \end{frame}

    \begin{frame}微服务
        一个比较新的概念——云函数
        IaaS, PaaS, SaaS, FaaS(函数即服务)
        特点:
        \begin{itemize}
        	\item 事件驱动
        	\item 无需管理服务器等基础设施
        	\item 弹性、可靠
        \end{itemize}        
    \end{frame}

    \begin{frame}
        亚马逊lambdas与容器、虚拟机、传统物理机的对比
        \begin{figure}
        	\centering
            \includegraphics[height=1.5in]{1.png}
        \end{figure}
        
    \end{frame}

    \begin{frame}应用场景
    	\begin{itemize}
    		\item 突发性的、计算密集的负载(如社交软件中的视频、图片的处理)
    		\item 后端:可实现无服务器后端,无需考虑弹性扩展、备份冗余,降低运维成本
    		\item 周期性/计划性的庞大数据量的处理
    		
    	\end{itemize}
    
    \end{frame}

    \begin{frame}
        \centering Thank you!
    \end{frame}

\end{document}